\documentclass[a4paper, 10pt, french]{article}

\usepackage[utf8]{inputenc}
\usepackage[T1]{fontenc}
\usepackage[frenchb]{babel}
\usepackage{lmodern}
\usepackage[autolanguage]{numprint}
\usepackage{enumitem}
\usepackage{array}
\usepackage{tabularx} \newcolumntype{C}{>{\centering}X}
\usepackage{multirow}
\usepackage{hhline}
\usepackage{collcell}
\usepackage{subcaption}
\usepackage[stable]{footmisc}

\usepackage[margin=2.5cm]{geometry}
\usepackage{multicol}
\usepackage[10pt]{moresize}
\usepackage{pdflscape}


\usepackage{amsthm}
\usepackage{amsmath}
\usepackage{amssymb}
\usepackage{mathrsfs}
\usepackage{amsopn}
\usepackage{stmaryrd}

\DeclareCaptionLabelFormat{listing}{Listing #2}
\DeclareCaptionSubType*[arabic]{table}
\captionsetup[subtable]{labelformat=simple}

\usepackage[langlinenos=true,newfloat=true]{minted}
\newcommand{\source}[5]{
  \begin{table}[H]
    \centering
    \small
\inputminted[frame=lines,linenos,style=colorful,fontfamily=tt,breaklines,
autogobble,firstline=#3,firstnumber=#3,lastline=#4,label={#2[#3--#4]}]{#1
}{../#2}
    
\captionsetup{name=Listing,labelformat=listing,labelsep=endash,labelfont={sc}}
    \caption{#5}
  \end{table}
  }

\newcommand{\codeC}[1]{\mintinline[style=colorful,fontfamily=tt]{C}{#1}}
\newcommand{\codeASM}[1]{\mintinline[style=colorful,fontfamily=tt]{nasm}{#1}}
\newcommand{\code}[1]{\texttt{#1}}
\newcommand{\foreign}[1]{\emph{#1}}

\newcommand{\mariobros}{\foreign{MarioBrOS}}



\title{Projet Rock'n'Roll : \mariobros}
\author{Florentin \bsc{Guth} \and Lionel \bsc{Zoubritzky}}

\begin{document}

\maketitle


\tableofcontents

\clearpage

% Example
%
% On a du code \codeC{int main() { return 0; }}
% 
% 
% \source{C}{ata\string_pio.c}{0}{20}{Disque}
% 
% \source{nasm}{loader.s}{0}{50}{Loader}

\section{Organisation générale}

\subsection{\foreign{Makefile}}

Les principales cibles de compilations sont les suivantes :
\begin{itemize}
 \item \code{make}, \code{make disk}, \code{make diskq} : crée une image disque 
et lance \foreign{QEMU} ;
 \item \code{make diskb} : crée une image disque et lance \foreign{Bochs} ;
 \item \code{make clean} : supprime tous les fichiers générés.
\end{itemize}

La compilation (pour le disque) s'effectue en plusieurs étapes :
\begin{itemize}
 \item créer une image disque vide \foreign{EXT2},
 \item installer \foreign{GRUB} dessus,
 \item y copier \code{kernel.elf}, l'exécutable du \foreign{kernel} une fois 
compilé,
 \item compiler les programmes utilisateurs (dossier \foreign{progs}) en 
exécutables et les copier,
 \item lancer l'émulateur.
\end{itemize}



\subsection{\foreign{Boot}}

\mariobros\ fonctionne aussi bien sous \foreign{Bochs} que \foreign{QEMU}. Le 
fonctionnement du \foreign{boot} est assez simple :
\begin{itemize}
 \item \foreign{GRUB} est d'abord lancé et détecte notre noyau 
\code{kernel.elf},
 \item on rentre alors dans le fichier \code{loader.s} qui permet de récupérer 
la taille de la mémoire,
 \item on lance ensuite la fonction \code{kmain} qui initialise tous les 
composants et rentre dans la boucle principale.
\end{itemize}


\clearpage
\section{Système de fichiers}

	Le système de fichier implémenté est EXT2.
	
	\subsection{Côté hardware}
	
		Les fonctions \verb|readPIO|, \verb|readLBA| et \verb|writeLBA| 
permettent d'accéder au disque dur en utilisant le système d'adressage LBA 28 
bits.
		
		Nous avons remarqué que l'émulateur \foreign{Bochs} n'allumait 
pas aux moments voulus le bit DRQ du port de contrôle ATA, qui signifie que le 
disque est prêt à effectuer des opérations IO, contrairement à ce que la 
spécification demandait. Par conséquent, la fonction \verb|poll| qui attend que 
ce bit soit allumé pour continuer à effectuer les opérations de lecture / 
écriture a été modifiée pour fonctionner sous \foreign{Bochs}.
	
	\subsection{Côté interface utilisateur}
	
		Les fonctions \verb|openfile|, \verb|read|, \verb|write| et 
\verb|close| de l'interface POSIX ont étés implémentées, avec la syntaxe et les 
flags présentés dans la bibliothèque Unix de OCaml. Les permissions ne sont pas 
vérifiées, seul le mode d'ouverture (\verb|O_RDONLY|, \verb|O_WRONLY|, 
\verb|O_APPEND|, etc\ldots) importe.
		
		Les fonctions du \foreign{shell} \verb|ls|, \verb|mkdir|, 
\verb|rm| sont aussi implémentées. \verb|rm| peut supprimer récursivement des 
dossiers sans risque de débordement de pile pour une profondeur quelconque de 
sous-dossiers.
		
	\subsection{Implémentation et difficultés rencontrées}
	
		La structure des inodes et des blocs est dictée par le standard 
de EXT2. Les champs de ces structures sont maintenus à jour de façon à ce que le 
disque puisse être monté sur un autre système d'exploitation et refléter les 
opérations réalisées sous \mariobros, à l'exception des champs concernant les 
dernières dates de modification, création, etc. qui ne sont pas actualisés.
		
		Les données des fichiers en EXT2 sont contenues dans des blocs, 
pointés par l'inode soit directement (douze \foreign{direct block pointers}), 
soit par une simple indirection (un \foreign{simple indirect block pointer} qui 
pointe vers un bloc d'adresses de blocs de données), soit par une double 
indirection, soit par une triple. L'implémentation permet en théorie de 
manipuler tous les fichiers de ce type, mais du fait de la taille du disque 
employé (64 Mo), nous n'avons pas pu tester la manipulation effective de gros 
fichiers.\\
		
		Les \foreign{file descriptors} (\foreign{fd}) ont été conçus 
pour répondre à trois contraintes : pouvoir effectuer la manipulation effective 
de fichiers, ne pas être réutilisables après avoir été fermés et ne pas être 
limités en nombre (autrement que par la taille de la mémoire physique). 
		
		Pour cela, l'OS dispose d'une \foreign{file descriptor table} 
(\foreign{fdt}) qui est un tableau dynamique de \foreign{fd}. Chaque 
\foreign{fd} est lui-même un pointeur alloué par l'OS vers un entier, cet entier 
servant d'index dans la \foreign{fdt}. Chaque entrée de la \foreign{fdt} est 
constituée d'un inode, de la taille du fichier, de la position du curseur du 
\foreign{fd} et de la valeur du \foreign{fd} pointant sur l'index de cette 
entrée.
		
		Pour dilater la \foreign{fdt}, il suffit d'allouer un bloc du 
double de sa taille et de recopier dedans la \foreign{fdt} actuelle. Pour la 
contracter, ce qui se produit lorsqu'elle est occupée pour moins d'un quart, un 
nouveau bloc est alloué de taille moitié de la \foreign{fdt} actuelle, et les 
entrées sont recopiées dans la nouvelle. Si une entrée doit changer d'index, le 
\foreign{fd} correspondant (qui est un de ses champs) est modifié en changeant 
l'index sur lequel il pointe.
		
		Enfin, à la fermeture d'un \foreign{fd}, celui-ci est libéré de 
la mémoire et la valeur du champ \foreign{fd} correspondant à son index dans la 
\foreign{fdt} est mis à 0, qui ne correspond à aucun pointeur valide, et donc à 
aucun \foreign{fd} valide. À chaque appel à une fonction comme \verb|read|, 
\verb|write|, etc., la correspondance entre le \foreign{fd} donné et le champ 
\foreign{fd} dans la \foreign{fdt} correspondant à son index est vérifiée, et la 
fonction considère le \foreign{fd} invalide si ce n'est pas le cas. Cette 
sécurité n'est cependant pas complète : si le pointeur \foreign{fd} est 
ré-alloué par la suite par \verb|openfile|, il pourra alors resservir.


\clearpage
\section{Gestion de la mémoire}

\subsection{Organisation de la mémoire}

La mémoire (virtuelle) se décompose comme suit :
\begin{table}[H]
 \centering
 \begin{subtable}[h]{.5\linewidth}
 \centering
 \begin{tabular}{|c|}
 \hline
 $\vdots$ \\
 Tas du \foreign{kernel} \\
 \hline
 Pile du \foreign{kernel} (une page, soit 4KB) \\
 Segment \foreign{data} \\
 Code du \foreign{kernel} \\
 \hline
 Espace réservé pour le \foreign{BIOS}, \foreign{GRUB} \\
 \hline
\end{tabular}
\caption{Espace mémoire du noyau}
\end{subtable}


\begin{subtable}[h]{.5\linewidth}
 \centering
 \begin{tabular}{|c|}
  \hline
 \foreign{Data} utilisateur \\
 Code utilisateur \\
 \hline
 Pile utilisateur \\
 $\vdots$ \\
 \hline
 $\vdots$ \\
 Tas de l'utilisateur \\
 \hline
 Pile du \foreign{kernel} (une page, soit 4KB) \\
 Segment \foreign{data} \\
 Code du \foreign{kernel} \\
 \hline
 Espace réservé pour le \foreign{BIOS}, \foreign{GRUB} \\
 \hline
 \end{tabular}

 \caption{Espace mémoire de l'utilisateur}
\end{subtable}
\caption{Organisation de la mémoire virtuelle}
\end{table}

La segmentation est utilisée en modèle \foreign{flat}, c'est-à-dire que tous les 
segments vont de 0 à 4GB. Celle-ci ne sert qu'à gérer le \foreign{user-mode}.


\subsection{\foreign{Paging}}

En ce qui concerne le \foreign{paging}, tout de 1MB à la fin de la pile du 
\foreign{kernel} est \foreign{identity-mapped} afin de pouvoir activer le 
\foreign{paging} de manière transparente (et ce qui permet aux processus de 
pouvoir exécuter du code noyau de manière transparente également), la page 0 
n'étant pas \foreign{mappée} pour détecter les erreurs.

Lorsqu'un utilisateur demande l'accès à une page virtuelle, on cherche la 
première page physique libre et on \foreign{mappe} la première vers la 
dernière. Ainsi, cela permet au tas de grandir automatiquement. Il en va de 
même pour la pile de l'utilisateur : lorsqu'il effectue un \foreign{Page 
Fault} pour cause de \foreign{Stack Overflow}, on peut \foreign{mapper} la page 
manquante et recommencer (désactivé pour faciliter le \foreign{debug}).

Il est également possible pour le \foreign{kernel} de demander l'accès à une 
page physique, ce qui est souvent utile : par exemple, pour charger le code 
d'un processus en mémoire (alors que l'on est dans le \foreign{page directory} 
du \foreign{kernel}), on demande l'accès aux pages physiques correspondant aux 
adresses virtuelles \code{0xFFFF0000 - 0xFFFFFFFF} du processus (le code 
utilisateur étant \foreign{linké} à l'adresse \code{0xFFFF0000}) et on y copie 
alors le code voulu.

\subsection{Tas}

Le tas (\code{mem\string_alloc} et \code{mem\string_free}) suit une structure 
de blocs standard. Initialement, il s'agit d'un seul bloc libre de la taille 
d'une page, et lorque l'on demande des tailles plus grandes, celui s'agrandit 
en demandant l'accès aux pages virtuelles contigües.

Les blocs libres sont doublement chaînés entre eux pour faciliter l'allocation. 
Lors de la libération d'un bloc, celui-ci est éventuellement fusionné avec ses 
voisins (physiques) si ceux-ci sont libres, afin d'éviter la fragmentation de 
la mémoire.

\source{C}{src/malloc.c}{21}{52}{Structure du tas}


\clearpage
\section{\foreign{Multitasking}}

\subsection{Processus}

La structure suivie est similaire à celle du piconoyau présenté lors du premier 
cours, en ceci que l'on retient pour chaque processus ses registres, son 
\foreign{page directory} et l'état de son tas.

Un ordonnanceur va alors sélectionner les processus selon leur priorité, à 
chaque tick de l'horloge (interruption \foreign{timer}), ou après un appel 
système qui laisse le processus courant en attente.

On dispose de l'appel système \code{hlt} (effectué par le processus \code{idle} 
en continu) qui permet au système d'être toujours sensible aux interruptions. 
Pour cela, on ignore les interruptions \foreign{timer} lorsque l'on est en 
train de répondre à un appel système, tandis que tout autre interruption 
provoque l'arrêt de la boucle d'attente et le retour à l'ordonnanceur, qui 
sélectionne un nouveau processus, etc\ldots.

\subsection{\foreign{Shell}}

\`A l'heure où nous mettons sous presse, le \foreign{shell} n'est pas encore un 
processus utilisateur à part entière. Nous espérons cependant qu'il en sera un 
le jour de la présentation.

Le \foreign{shell} premet d'éxécuter des commandes (la liste étant accessible 
avec \code{help}), dispose d'un curseur (qui n'est pas visible depuis que l'on 
\foreign{boot} sur le disque pour une raison inconnue) et d'un historique. En 
tant que processus, toutes les commandes ou presque utilisent des appels 
systèmes.

\subsection{La commande \code{run}}

Cette commande permet de lancer un exécutable \foreign{ELF} situé dans le 
dossier \code{progs} (\code{run foo} va ainsi lancer le programme 
\code{progs/foo.elf}). Ces exécutables correspondent à un ficher \foreign{C} 
situé dans \code{progs/src} qui a été compilé à l'aide d'une bibliothèque pour 
pouvoir utiliser les appels systèmes.

\source{nasm}{progs/src/lib.s}{10}{21}{Exemple d'appel système (côté 
utilisateur)}


\clearpage
\section{Bugs et améliorations possibles}

Il y a certainement quelques fuites de mémoire de temps à autre, notamment au 
niveau du \foreign{shell} (c'est plus faute de temps qu'autre chose). De plus, 
plus de gestion des erreurs (mauvaises commandes, manque de mémoire, \ldots) 
serait nécessaire.

Pouvoir exécuter le \foreign{shell} comme un processus aurait été un énorme 
plus, avec des \foreign{buffers} d'entrée, de sortie, d'erreur et les 
opérateurs \foreign{pipe}, les indirections\ldots

Les appels systèmes \code{new\string_channel}, \code{send} et \code{receive} 
sont codés (depuis le piconoyau) mais pas portés dans \mariobros, encore une 
fois par manque de temps.

Enfin, dans l'état actuel des choses, appeler la commander \code{run} ne rend 
pas la main au \code{shell}\ldots C'est bizarre que la première commande marche 
sans la seconde, mais c'est un problème qui serait réglé par le fait que le 
\foreign{shell} soit un processus à part enitère.

\end{document}
