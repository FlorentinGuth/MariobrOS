\documentclass[a4paper, 10pt, french]{article}

\usepackage[utf8]{inputenc}
\usepackage[T1]{fontenc}
\usepackage[frenchb]{babel}
\usepackage{lmodern}
\usepackage[autolanguage]{numprint}
\usepackage{enumitem}
\usepackage{array}
\usepackage{tabularx} \newcolumntype{C}{>{\centering}X}
\usepackage{multirow}
\usepackage{hhline}
\usepackage{collcell}
\usepackage{subcaption}
\usepackage[stable]{footmisc}

\usepackage[margin=2.5cm]{geometry}
\usepackage{multicol}
\usepackage[10pt]{moresize}
\usepackage{pdflscape}


\usepackage{amsthm}
\usepackage{amsmath}
\usepackage{amssymb}
\usepackage{mathrsfs}
\usepackage{amsopn}
\usepackage{stmaryrd}

\DeclareCaptionLabelFormat{listing}{Listing #2}
\DeclareCaptionSubType*[arabic]{table}
\captionsetup[subtable]{labelformat=simple}

\usepackage[langlinenos=true,newfloat=true]{minted}
\newcommand{\source}[5]{
  \begin{table}[H]
    \centering
    \inputminted[frame=lines,linenos,style=colorful,fontfamily=tt,breaklines,autogobble,firstline=#3,firstnumber=#3,lastline=#4,label={#2[#3--#4]}]{#1}{../src/#2}
    \captionsetup{name=Listing,labelformat=listing,labelsep=endash,labelfont={sc}}
    \caption{#5}
  \end{table}
  }

\newcommand{\codeC}[1]{\mintinline[style=colorful,fontfamily=tt]{C}{#1}}
\newcommand{\codeASM}[1]{\mintinline[style=colorful,fontfamily=tt]{nasm}{#1}}
\newcommand{\code}[1]{\texttt{#1}}
\newcommand{\foreign}[1]{\emph{#1}}

\newcommand{\mariobros}{\foreign{MarioBrOS}}



\title{Projet Rock'n'Roll : \mariobros}
\author{Florentin \bsc{Guth} \and Lionel \bsc{Zoubritzky}}

\begin{document}

\maketitle


\tableofcontents

\clearpage

% Example
%
% On a du code \codeC{int main() { return 0; }}
% 
% 
% \source{C}{ata\string_pio.c}{0}{20}{Disque}
% 
% \source{nasm}{loader.s}{0}{50}{Loader}

\section{Organisation générale}

\subsection{\foreign{Makefile}}

Les principales cibles de compilations sont les suivantes :
\begin{itemize}
 \item \code{make}, \code{make disk}, \code{make diskq} : crée une image disque et lance \foreign{QEMU} ;
 \item \code{make diskb} : crée une image disque et lance \foreign{Bochs} ;
 \item \code{make run}, \code{make runq} : crée une image \foreign{ISO} et lance \foreign{QEMU} ;
 \item \code{make runb} : crée une image \foreign{ISO} et lance \foreign{Bochs} ;
 \item \code{make clean} : supprime tous les fichiers générés.
\end{itemize}

La compilation (pour le disque) s'effectue en plusieurs étapes :
\begin{itemize}
 \item créer une image disque vide \foreign{EXT2},
 \item installer \foreign{GRUB} dessus,
 \item y copier \code{kernel.elf}, l'exécutable du \foreign{kernel} une fois compilé,
 \item compiler les programmes utilisateurs (dossier \foreign{progs}) en exécutables et les copier,
 \item lancer l'émulateur.
\end{itemize}



\subsection{\foreign{Boot}}

\mariobros fonctionne aussi bien sous \foreign{Bochs} que \foreign{QEMU}. Le fonctionnement de


\section{Système de fichiers}

\section{Gestion de la mémoire}

\section{\foreign{Multitasking}}


\end{document}
